\documentclass[11pt]{article}

% Packages
\usepackage{fontspec}
\usepackage{unicode-math}
%\usepackage[utf8]{inputenc}
\usepackage{minted}
\usepackage{xcolor}
\usepackage{hyperref}
\usepackage{amsmath,amsthm}

% Minted setup voor Lean 4
\usemintedstyle{default}
\setminted{
    breaklines=true,
    fontsize=\small,
    frame=lines,
    framesep=2mm,
    baselinestretch=1.2,
    bgcolor=lightgray!10,
    linenos=false
}

% Theorem environments
\newtheorem{theorem}{Theorem}
\newtheorem{lemma}[theorem]{Lemma}
\newtheorem{corollary}[theorem]{Corollary}
\newtheorem{definition}[theorem]{Definition}
\newtheorem{example}[theorem]{Example}
\newtheorem{proposition}[theorem]{Proposition}

% Custom commands
\newcommand{\Roperator}{\mathsf{R}}
\newcommand{\Ind}{\mathsf{Ind}}
\newcommand{\CRB}{\mathsf{CRB}}

\title{A Formalization of Lewis's Convention Theory\\ and Critique of Sillari's Modal Analysis}
\author{Your Name\\
Your Institution\\
\texttt{your.email@institution.edu}}
\date{\today}

\begin{document}

\maketitle

\begin{abstract}
We present a complete formalization in Lean 4 of David Lewis's theory of conventions 
and common knowledge, along with a critical analysis of Sillari's attempted modal 
logic formalization. Our development consists of three main components: (1) a 
formalization of Lewis's original argument using a justification logic framework 
with explicit reasons, (2) an implementation of Sillari's Kripke semantics approach, 
and (3) formal proofs showing fundamental problems with Sillari's formalization. 
Specifically, we demonstrate that crucial axioms (B3/A1 and C4) fail under Sillari's 
definitions, and that Lewis's main theorem either becomes false or trivial depending 
on how it is interpreted. Our formalization comprises approximately [X] lines of 
Lean code and includes [Y] theorems and lemmas.
\end{abstract}

\section{Introduction}

David Lewis's 1969 theory of conventions provides a foundational account of how 
coordination problems are solved through common knowledge. His framework has been 
influential in philosophy, game theory, linguistics, and social epistemology. 
However, the precise logical structure of Lewis's argument has been debated.

Sillari (2005, 2008) attempted to formalize Lewis's account using standard modal 
logic with Kripke semantics. While this approach has pedagogical appeal, we show 
through formal verification in Lean 4 that it suffers from fundamental problems:

\begin{itemize}
\item Lewis's crucial axiom A1 (Sillari's B3) fails under the Kripke semantics
\item Cubitt and Sugden's axiom C4 also fails
\item Lewis's main theorem either has counterexamples or becomes vacuous
\end{itemize}

We also provide an alternative formalization using justification logic with explicit 
reasons to believe, showing that Lewis's original argument can be made rigorous 
without the problems that plague Sillari's approach.

\subsection{Contributions}

\begin{itemize}
\item First complete formalization of Lewis's convention argument in a proof assistant
\item Formal verification that Sillari's modal logic formalization is flawed
\item Explicit counterexamples demonstrating the failure of key axioms
\item Alternative formalization using justification logic
\item Machine-checked proofs of all claims
\end{itemize}

\subsection{Related Work}

Prior work includes Cubitt and Sugden's (2003) informal analysis, Sillari's modal 
formalization, and various game-theoretic treatments. To our knowledge, this is 
the first mechanized verification of these results.

\section{Lewis's Original Framework}

Lewis's account centers on two key concepts: \emph{reasoning} ($\Roperator$) and 
\emph{indication} ($\Ind$).

\subsection{Core Definitions}

\begin{definition}[Reasoning]
$\Roperator_i \phi$ means individual $i$ has reason to believe proposition $\phi$.
\end{definition}

\begin{definition}[Indication]
$\Ind_A^i \psi$ means that given common knowledge $A$, individual $i$'s reasons 
for $A$ indicate $\psi$.
\end{definition}

In Lean, we represent these using explicit reason terms:

\begin{minted}{lean}
-- R: having a reason to believe
def R (rb : reason → indiv → Prop → Prop) (i : indiv) (φ : Prop) : Prop :=
  ∃ r, rb r i φ

-- Indication: having reason to believe the implication
def Ind (rb : reason → indiv → Prop → Prop) 
    (φ : Prop) (i : indiv) (ψ : Prop) : Prop :=
  R rb i (φ → ψ)
\end{minted}

\subsection{Justification Logic Framework}

Our formalization uses a logic of reasons based on Artemov's justification logic. 
The key axiom is the \emph{application rule}:

\begin{minted}{lean}
-- Application rule: if i has reason s for (α → β) 
-- and reason t for α, then i has reason (s * t) for β
axiom AR (rb : reason → indiv → Prop → Prop) :
  ∀ {s t : reason} {i : indiv} {α β : Prop},
  rb s i (α → β) → rb t i α → rb (s * t) i β
\end{minted}

This allows us to track \emph{how} individuals reason, not just \emph{that} they 
have beliefs.

\subsection{The R-Closure}

Lewis's main construction is the R-closure: the set of propositions reachable 
from $\phi$ by iteratively applying reasoning operators.

\begin{minted}{lean}
inductive RC (R : indiv → Prop → Prop) (φ : Prop) : Prop → Prop
| base : RC R φ φ
| step {u : Prop} (j : indiv) (hu : RC R φ u) : 
    RC R φ (R j u)
\end{minted}

This captures nested reasoning like: $\phi$, $\Roperator_j \phi$, 
$\Roperator_i (\Roperator_j \phi)$, etc.

\subsection{Lewis's Main Theorem}

\begin{theorem}[Lewis]
Under conditions C1-C4, if $p$ is in the R-closure of $\phi$, then every 
individual reasons to $p$.
\end{theorem}

\begin{proof}
The proof proceeds by induction on the R-closure structure:

\begin{minted}{lean}
lemma everyone_reason_of_rc
  (A1 : ∀ {i : indiv} {p : Prop}, Ind A i p → R i A → R i p)
  (A6 : ∀ {i j : indiv} {u : Prop}, 
    Ind A i (R j A) ∧ R i (Ind A j u) → Ind A i (R j u))
  (C1 : ∀ i : indiv, R i A)
  (C2 : ∀ i j : indiv, Ind A i (R j A))
  (C3 : ∀ i : indiv, Ind A i φ)
  (C4 : ∀ {i j : indiv} {u : Prop}, 
    Ind A i u → R i (Ind A j u))
  (hp : RC R φ p) : ∀ i : indiv, R i p := by
  intro i
  have hInd : Ind A i p := 
    ind_of_rc A6 C2 C3 C4 hp
  exact A1 hInd (C1 i)
\end{minted}

The key insight is that indication propagates through the closure: if $q$ is in 
the R-closure, then every individual indicates $q$ (lemma \texttt{ind\_of\_rc}).
\end{proof}

\subsection{Derived Results}

We verify specific instances showing elements at various depths in the R-closure:

\begin{minted}{lean}
-- Depth 1: R j φ
lemma L1 [...] (i j : indiv) : R i (R j φ)

-- Depth 2: R j (R k φ)  
lemma L2 [...] (i j k : indiv) : R i (R j (R k φ))

-- Depth 3: R j (R k (R ℓ φ))
lemma L3 [...] (i j k ℓ : indiv) : R i (R j (R k (R ℓ φ)))
\end{minted}

\section{Sillari's Modal Formalization}

Sillari attempts to formalize Lewis using standard Kripke semantics for epistemic 
logic. While conceptually simpler, we show this approach is fundamentally flawed.

\subsection{Kripke Semantics Approach}

\begin{minted}{lean}
-- Multi-agent Kripke frame
structure MultiAgentFrame (Agent : Type) where
  World : Type
  rel : Agent → World → World → Prop

-- R: knowledge operator (box modality)
def R (i : Agent) (φ : frame.World → Prop) : 
    frame.World → Prop :=
  fun w => ∀ v, frame.rel i w v → φ v

-- Ind: knowledge plus material implication
def Ind (i : Agent) (φ ψ : frame.World → Prop) : 
    frame.World → Prop :=
  fun w => R i φ w ∧ (φ w → ψ w)
\end{minted}

\subsection{Common Reason to Believe via Reachability}

\begin{minted}{lean}
-- Reachability via transitive closure
inductive trcl (r : frame.World → frame.World → Prop) : 
    frame.World → frame.World → Prop
| base {x y} : r x y → trcl r x y
| step {x y z} : r x y → trcl r y z → trcl r x z

-- CRB: common reason to believe
def CRB (ψ : frame.World → Prop) (s : frame.World) : Prop :=
  ∀ w, trcl connected s w → ψ w
\end{minted}

\section{Failure of Sillari's Formalization}

\subsection{Axiom B3 Fails (Lewis's A1)}

\begin{theorem}[B3 Counterexample]
There exists a Kripke frame where $\Roperator_i \phi \land \Ind_i^\phi \psi$ 
holds but $\Roperator_i \psi$ fails.
\end{theorem}

\begin{proof}
Consider a two-world frame with worlds $s$ and $t$, where agent $i$ relates 
$s$ to $t$ but not $s$ to itself. Let $\phi(w) := (w \neq s)$ and 
$\psi(w) := (w \neq t)$.

\begin{minted}{lean}
lemma B3_fails
  (h1 : two_worlds s t)
  (h2a : ¬ frame.rel i s s)
  (h2b : frame.rel i s t) :
    ¬ (∀ w (φ ψ : frame.World → Prop), 
       R i φ w → Ind i φ ψ w → R i ψ w) := by
  let ψ := fun w => w ≠ t
  let φ := fun w => w ≠ s
  push_neg
  -- R i φ s holds: t is the only successor, and φ t
  have h4 : R i φ s := by rw [R]; aesop
  -- Ind i φ ψ s holds
  have h5 : Ind i φ ψ s := by rw [Ind]; aesop
  -- But R i ψ s fails: at t, ¬ψ t
  have h6 : ¬ R i ψ s := by
    intro hR
    exact (hR t h2b) rfl
  refine ⟨s, φ, ψ, ?_, ?_, ?_⟩
  · exact h4
  · exact h5  
  · exact h6
\end{minted}

At world $s$: $\Roperator_i \phi$ holds (the only accessible world $t$ satisfies 
$\phi$), and $\Ind_i^\phi \psi$ holds (since $\phi \to \psi$ at $s$). But 
$\Roperator_i \psi$ fails because at accessible world $t$, we have $\neg\psi(t)$.
\end{proof}

This is devastating for Sillari's approach since Lewis's original argument 
crucially depends on axiom A1.

\subsection{Axiom C4 Also Fails}

Cubitt and Sugden proposed axiom C4 as essential for Lewis's account:

\begin{theorem}[C4 Counterexample]
$\Ind_i^\phi \psi \not\Rightarrow \Roperator_i(\Ind_j^\phi \psi)$
\end{theorem}

\begin{minted}{lean}
lemma C4_fails
  (h2a : ¬ frame.rel i s s)
  (h2b : frame.rel i s t) :
    ¬ ∀ w (φ ψ : frame.World → Prop), 
      (Ind i φ ψ w → R i (Ind j φ ψ) w) := by
  let φ := fun _ : frame.World => True
  let ψ := fun w : frame.World => w = s
  push_neg
  have h3 : Ind i φ ψ s := by
    constructor
    { intro w _; aesop }
    { aesop }
  have h3a : ¬ R i (Ind j φ ψ) s := by
    rw [R]
    push_neg
    use t
    constructor
    { exact h2b }
    { intro hn
      have hphi : φ t := by aesop
      have hp : ψ t := hn.2 hphi
      have h3b : ¬ ψ t := by aesop
      aesop }
  use s, φ, ψ
\end{minted}

\subsection{Lewis's Theorem: Two Failed Interpretations}

Sillari's formulation of Lewis's theorem is ambiguous. We examine both 
interpretations:

\subsubsection{Local Interpretation (FALSE)}

If assumptions C1-C3 hold only at world $s$:

\begin{theorem}[Counterexample: One Agent]
There exists a frame with one agent and three worlds where all local conditions 
hold but $\CRB$ fails.
\end{theorem}

\begin{minted}{lean}
lemma Lewis_fails_1i
  (h3w : three_worlds s u v)
  (hrel : frame.rel = fun (_ : Agent) (w1 w2 : frame.World) =>
    (w1 = s ∧ w2 = u) ∨ (w1 = u ∧ w2 = v)) :
  ∃ (φ : frame.World → Prop),
    R i1 φ s ∧
    Ind i1 (R i1 φ) (R i1 (R i1 φ)) s ∧  
    Ind i1 (R i1 φ) (R i1 (fun w => w = u)) s ∧
    ¬ CRB (fun w => w = u) s
\end{minted}

Frame structure: $s \to u \to v$ (linear chain). With $\phi := (\cdot \neq s)$ 
and $\psi := (\cdot = u)$, all local conditions hold at $s$, but $\CRB\ \psi\ s$ 
fails because $v$ is reachable and $\psi(v)$ is false.

\subsubsection{Global Interpretation (TRIVIAL)}

If assumptions hold at all worlds, the proof becomes vacuous:

\begin{minted}{lean}
lemma lewis_s_2
  (C1 : ∀ w, Rg φ w)
  (C3 : ∀ w, Ind i (Rg φ) (Rg ψ) w) :
    CRB ψ s := by
  have hRgψ_all : ∀ w, Rg ψ w := 
    fun w => (C3 w).2 (C1 w)
  intro v hv
  induction hv with
  | base h_edge =>
      rcases h_edge with ⟨j, hj⟩
      exact (hRgψ_all x) j y hj
  | step _ _ ih => exact ih
\end{minted}

Note: Assumption C2 is completely unused! This suggests the global interpretation 
misses Lewis's intended logical structure.

\section{Implementation Statistics}

Our formalization consists of three main files:

\begin{itemize}
\item \texttt{Neeley.lean}: Lewis's original argument (justification logic) - [X] lines
\item \texttt{Neeley\_closure.lean}: R-closure and inductive proofs - [Y] lines  
\item \texttt{Sillari.lean}: Kripke semantics and counterexamples - [Z] lines
\end{itemize}

Total statistics:
\begin{itemize}
\item Lines of code: [Total]
\item Definitions: [Count]
\item Theorems/Lemmas: [Count]
\item Axioms (justification logic): 4
\item Counterexamples: 4
\end{itemize}

\section{Discussion}

\subsection{Why Sillari's Approach Fails}

The fundamental problem is that Kripke semantics conflates \emph{having} a belief 
with \emph{having a reason} for that belief. Lewis's original argument tracks 
explicit reasons and their composition, which cannot be captured by simple 
accessibility relations.

The failure of B3/A1 shows that the material implication $\phi \to \psi$ at a 
world does not suffice to transmit reasons from $\phi$ to $\psi$. Lewis needs 
something stronger: a \emph{reason for the implication}.

\subsection{Advantages of Justification Logic}

Our alternative formalization using justification logic:
\begin{itemize}
\item Makes the reasoning structure explicit
\item Avoids the failed axioms (A1 and A6 become provable)
\item Matches Lewis's original informal presentation
\item Provides constructive proofs (not just semantic validity)
\end{itemize}

\subsection{Implications for Convention Theory}

These results have philosophical significance:
\begin{itemize}
\item Standard epistemic logic may be inadequate for analyzing conventions
\item The structure of common knowledge in coordination problems is more subtle 
than previously recognized
\item Formalization reveals hidden assumptions in informal arguments
\end{itemize}

\section{Conclusion}

We have presented the first complete mechanized verification of Lewis's convention 
theory and demonstrated fundamental flaws in Sillari's modal logic formalization. 
Our results show that:

\begin{enumerate}
\item Lewis's original argument can be made rigorous using justification logic
\item Sillari's Kripke semantics approach fails on multiple axioms
\item Machine verification reveals problems invisible to informal analysis
\end{enumerate}

Future work includes extending to full game-theoretic conventions, analyzing 
alternative modal approaches, and exploring other applications of justification 
logic to social epistemology.

\subsection{Code Availability}

Complete source code is available at: \url{https://github.com/yourusername/lewis-lean}

\bibliographystyle{plain}
\bibliography{references}

\end{document}